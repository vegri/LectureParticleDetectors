The kinetic energy of the electrons is proportional to $\frac{1}{\Delta E^2}$. $\Delta E$ is the
energy transfer of the projectile to the shell electron.

\[\frac{d\sigma}{d\Delta E} = \frac{2\pi z^2 \alpha^2 \hbar^2}{\beta^2 m_e}\cdot \frac{1}{\Delta
E^2}
\]

The tails of this distribution stretch to $\Delta E_\text{max}$,

\[\Delta E_\text{max} = \frac{2m_e c^2 \beta^2 \gamma^2}{1+ 2\gamma
\frac{m_e}{M}+\left( \frac{m_e}{M} \right)^2}\]

with the limits

\begin{itemize}
  \item $\gamma\rightarrow\infty$: $\Delta E \rightarrow \gamma Mc^2$
  \item $m_e \rightarrow M$: $\Delta E \rightarrow m_e c^2 (\gamma -1) = E- m c^2$ \\
  		The entire energy is transfered to the shell electron.
\end{itemize}

These tails can stretch very far for relativistic particles and produce electrons with energies of
several keV. They can be detected as so-called Delta electrons in detectors with high granularity
and spatial resolution.
\\
Rare irradation with high energy leads to more flucuations in $\frac{dE}{dx}$ measurements used to
identify particles and therefore worsens the resolution. Even with detectos of low granularity
does this emission of Delta electrons result in a diminishment of spatial resolution. 
\\
A precise calculation shows:

\[\Delta E (\Theta) = \frac{2m_e}{\text{tan}^2\Theta} \]

Larger angles of irradiation leads to lower energies of the Delta electrons which results in a
shorter range.
