To derive the Bethe formula in a classic way, we consider the energy loss $\frac{dE}{dx}$ of a heavy
(i.e. $m>>m_e$) charged particle scattering on a shell electron of a target atom.
\\
We make the following assumptions:

\begin{itemize}
  \item the shell electron is at rest (neglection of orbital movement and recoil),
  \item the energy transfer is much larger than the binding energy of a shell electron.
\end{itemize}

{\Huge DRAWING}

The momentum transfer is the time integral of the force on the target, caused by the electric field
of the projectile. For the longitudinal and transversal components of the electric field applies

\[E_l(-x)=-E_l(x)~~~~~~~~~~~~~~~~~~~E_t(-x)=E_t(x)\]

i.e. only the transversal component is of importance, as the longitudinal components of the momentum
transfer add up to zero. This yields

\[\Delta p= \int_{-\infty}^{\infty}F\cdot dt = \int_{-\infty}^{\infty}eE_t\cdot dt =
e\int_{-\infty}^{\infty}E_t\frac{dt}{dx}\cdot dx
=e\int_{-\infty}^{\infty}E_t\frac{1}{v}\cdot dx
=\frac{e}{v}\int_{-\infty}^{\infty}E_t\cdot dx\]

With the Gauß formula $\int_{-\infty}^{\infty}E_t2\pi bdx=4\pi ze$ follows

\[\Delta p = \frac{2ze^2}{vb}\]

which yields for the energy transfer

\[\Delta E=\frac{\Delta p^2}{2m_e}=\frac{2z^2e^4}{m_ev^2b^2} .\]

With an electron densitiy of $n_e$, the energy loss results in 

\[-dE(b)=\Delta E(b)n_edv=\frac{2z^2e^4}{m_ev^2b^2}n_e2\pi b db dx\]

After integration from $b_min$ to $b_max$ we obtain:

\[-\left(\frac{dE}{dx}\right)=\frac{4\pi z^2
e^4}{m_ev^2}n_e\text{ln}\left(\frac{b_{\text{max}}}{b_{\text{min}}}\right)\]

Now we have to estimate $b_{\text{min}}$ and $b_{\text{max}}$. $b_{\text{min}}$ we can estimate with
the help of the kinematic limit: A head-on collision yields the greatest possible energy transfer

\[\Delta E_{\text{max}}=\frac{1}{2}m_e(2v)^2\gamma^2.\]

With the obtained relation 

\[\Delta E(b)=\frac{2z^2e^4}{m_ev^2b^2}\overset{!}{=}\Delta E_{\text{max}}\] 

we get

\[b_{\text{min}}=\frac{ze^2}{\gamma m_e v^2}.\]

The estimation of $b_{\text{max}}$ follows from the ``adiabatic invariance'': The target electrons
are bound in atoms and orbit the nucleus with a mean orbital frequency $\overline{\nu}$. The
duration of the disturbance $\Delta t$ has to be shorter than the period $\tau$ for a energy
transfer to happen: 

\[\Delta t=\frac{b}{\gamma v} \le \tau =\frac{1}{\overline{\nu}}\]

This yields 

\[b_{\text{max}}=\frac{\gamma v}{\overline{\nu}}.\]

Now we introduce a quantity for the electron density of the target material:

\[n_e=N_A\cdot \rho\cdot \frac{Z}{A}\]

with the Avogadro number $N_A$, the target density $\rho$, the atomic number $Z$ and mass number
$A$. Inserting the limits for the impact parameter into the formula and substituting $n_e$ leads to 

\[-\left(\frac{dE}{dx}\right)_{\text{coll}} = \frac{4\pi z^2e^4}{m_ev^2}N_A\cdot \rho
\frac{Z}{A}\cdot\text{ln}\left(\frac{\gamma^2 m_e v^3}{2e^2\overline{\nu}}\right), \]

which matches the classical formula of Bohr. This describes the loss of energy of heavy particles
(protons, $\alpha$ particles, \ldots) through excitation and ionisation. For lightweight particles
quantum effects have to be considered.
\\
A quantum mechanical calculation leads to Bethe-Bloch(-Sternheimer) formula:

\[-\left(\frac{dE}{dx}\right)_{\text{coll}} = 2\pi N_A r_e^2 m_e c^2 \rho \frac{Z}{A}
\frac{z^2}{\beta^2}\left[ \text{ln} \left( \frac{2m_e c^2 \gamma^2 \beta^2 W_{\text{max}}}{I^2}
\right) -2\beta^2 -\delta -2\frac{c}{z} \right]\]

with

\[\beta =
\frac{v}{c},~~~~~~~~~~~~\gamma=\frac{1}{\sqrt{1-\beta^2}}~~~~~~~~~~~~
r_e=\frac{1}{4\pi\epsilon_0}\cdot\frac{e^2}{m_e c^2}\]

as well as

sowie
\begin{description}
\item[$z$]charge of the incoming particle
\item[$Z, A$] atomic and mass number of the target
\item[$\rho$] target density
\item[$N_A$] Avogadro number
\item[$I$] mean ionisation potential (material constant of the target)
\item[$W_{\text{max}}$] max. energy transfer in one collision
\item[$\delta$] correction of density (polarisation effects, $\delta \approx 2\text{ln}(\gamma)+K$)
\item[$c$] ``shell correction'' (essential for small velocities of projectiles )
\end{description}

Remarks to Bethe-Bloch formula:

\begin{itemize}
  \item it corresponds well to the loss of energy in the scope of $0,1 < \gamma\beta < 100$;
  \item there are three scopes:
  			\begin{itemize}
  			  \item for low energies, there is a decline down to a minimum (bei $\gamma\beta$
  			  ca. 3-3,5), particles at this point are minimal ionising particles (MIP);
  			  \item after that, a logarithmic incline can be observed for increasing particle energy, the
  			  so-called ``relativistic incline'';
  			  \item for high energies the Fermi plateau is reached: the energy loss approaches a
  			  saturation point caused by polarisation effects (correction of density);
  			  \end{itemize}
  \item the energy loss is a statistical process.
\end{itemize}

The Bethe-Bloch formula describes the mean loss of energy through ionisation and excitation. It
applies to all charged particles except for electrons and positrons. For these, we have to consider
the identic masses and indistinguishability of the collision participants. Our derivation differs
numerically from the Bethe-Bloch formula in a factor of 2, which is caused by a lack of
consideration of distant collisions. There are different variations in the formula of the quantum
mechanical description of the loss of energy $\frac{dE}{dx}$. This is a result of different 
parameterisation of distant collisions, i.e. a loss of energy in which the bond of electrons in the
atomic shell is non-negligible.
\\
Usually, the loss of energy per distance $\frac{1}{\rho}\frac{dE}{dx}$ is given where $\rho$ is the
density in $\frac{\text{g}}{\text{cm}^3}$. $\frac{1}{\rho}\frac{dE}{dx}$ for MIPs is only weakly
dependant of the absorber material und amounts to ca.

\[2 \text{MeV}\frac{\text{cm}^2}{\text{g}}.\]
