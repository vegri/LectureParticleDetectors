Even without external electric fields, free charge carriers are in motion due to their thermal
energy. This kind of movement is described by a Maxwell distribution. For ions, we have

\[\langle E_{\text{kin}} \rangle = \frac{1}{2} m \langle u^2 \rangle ~~~~~\text{mit}~~~~ \langle u^2
\rangle = \frac{3kT}{m}
\]

We start with a punctiform charge distribution which then transforms into a \ldots

\begin{figure}[H]
	\centering
	\includegraphics[width=0.5\textwidth]{dummy.jpg}
\end{figure}

This process can be described with 

\[\frac{dN}{N} = \frac{1}{\sqrt{4\pi\cdot D\cdot t}}\, \text{exp}\left(-\frac{x^2}{4D\cdot
t}\right)\,\mathrm{d}x
\]

whereas the width $\sigma_x=\sqrt{2D\cdot t}$ is dependent of the diffusion coefficient $D$. The
faster the particles are, the larger $D$ gets. In particular, $\langle u^2 \rangle \sim
\frac{1}{m}$ increases with decreasing mass of the particle.
\\
The mean free path in the diffusion process is

\[\lambda(E_{\text{kin}}) = \frac{1}{\frac{N_0\cdot\rho}{A}\cdot \sigma(E_{\text{kin}})}, \]

therefore a function of the kinetic energy of the charged particle.
Previously, the width of the Gaussian distribution applied to diffusion in one dimension. In three
dimension, it is

\[\sigma = \sqrt{6D\cdot t} . \]

The diffusion coefficient can be calculated with the help of kinetic gas theory:

\[D=\frac{1}{3} \langle u \rangle \cdot \lambda~~~~~\text{mit}~~~~\langle u \rangle
=\sqrt{\frac{8kT}{\pi m}}\]

as well as $\lambda$ as mean free path of the charge carrier. That way, we obtain the following
explicit dependencies for $D$:

\begin{itemize}
  \item $D\sim 1/\sqrt{m}$
  \item $D\sim \sqrt{T^3}$
  \item $D\sim 1/\rho$
\end{itemize}

Here, $\sigma_0$ is the total cross section of the impact of the charge carrier with a gas molecule.
