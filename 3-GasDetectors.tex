The relevant processes for loss of energy in gas detectors are excitation and ionisation. 
\\
The excitation of an atom $A$ through a charged particle $x$ via

\[x+A \longrightarrow x+ A^*\]

needs an exact amount of energy in the transfer. Typical cross sections are in the scope of
$\sigma_{\text{Anregung}}=10^{-17}\text{cm}^2$. The excitation energy can be emitted through the
following processes:

\begin{itemize}
  \item Radiation:\\ $A^*\longrightarrow A+h\nu$
  \item Collision:\\ z.B. $\text{Ne}^*+\text{Ar}\longrightarrow \text{Ne}+\text{Ar}^++e^-$
  \item Ion-molecule forming (noble gases):\\ $\text{He}+\text{He} \longrightarrow\text{He}^+_2+e^-$
\end{itemize}

The ionisation of an atom $A$ via

\[x+A \longrightarrow x+ A^+ +e^- \]

however, does not need an exact amount of energy. Typical cross sections are in the scope of
$\sigma_{\text{Ion}}=10^{-16}\text{cm}^2$, which is larger than $\sigma_{\text{Anregung}}$. But as a
large amount of energy is necessary to ionise an atom while transfers with smaller energy are more
frequent, excitation dominates over ionisation.
\\
There are two kinds of ionisation: 

\begin{itemize}
  \item primary ionisation:\\ $x+A \longrightarrow x+A^++e^-$
  \item secondary ionisation:\\ $x+A \longrightarrow x+A^++\delta_{e^-}$ \\ $\delta_{e^-}+A
  \longrightarrow x+A^++e^-$
\end{itemize}

